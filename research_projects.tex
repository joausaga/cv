% !TEX root = jorge_saldivar.tex

\section{Research\\Projects}

\begin{project}
\title{Supporting Proactive Diabetes Screenings to Improve Health Outcomes}
\supervisor{Building predictive models to help identify patients at risk of developing diabetes}
\description{
	- Research on the effectiveness of the diabetes screening guidelines currently being used in the United States\\
	- Extract and transform raw medical data into tidy data ready to be loaded into a clean ER data scheme\\ 
	- Development and selection of features to be use in the predictive models\\
	- Implementation, validation, and evaluation of predictive models using Decision Tress, Random Forest, and Support Vector Machine algorithms
}
\end{project}
\begin{project}
  \title{Participa}
  \supervisor{Promoting civic participation in the innovation of public services and policies}
  %\duration{}
  \description{
  	%- Coordinated the implementation of an analytics tool that uses machine learning (ML) and natural language processing (NLP) techniques (classification, clustering, concept extraction, sentiment analysis) to automate the process of crowdsourced textual data\\
	- Designed and implemented a tool that integrates crowdsourcing idea technologies with general purpose social networks, like Facebook\\%\footnote{Code repository: \url{https://github.com/joausaga/social-ideation}}\\ 
	- Implemented a K-means algorithm to cluster similar crowdsourced civic contributions\\
	- Applied a K-means algorithm to discover patterns in the collective behavior of online innovation
communities\\
	- Contributed to the execution of a real-case process of participatory public service innovation in Asunción, the capital city of Paraguay ($\sim$ 200 participants)\\
	%- Designed machine learning and natural language processing algorithms to analyze crowdsourced civic input\\%\footnote{Code repository: \url{https://github.com/ParticipaPY/civic-crowdanalytics}}\\
	- \cusemph{Co-authored four scientific publications on the topics of Civic Technologies, Crowdsourcing, Virtual Reality, and Collective Intelligence}\\\\
	%- Performed a systematic review of the literature following the PRISMA methodology on technologies proposed to foster civic participation in social innovation processes\\
	\footnotesize{\textcolor{mygray}{\textbf{Research methods applied:} case studies - surveys - participatory design workshops - interviews - SLR}}
  } 
\end{project}
\begin{project}
  \title{The Finnish Experiment}
  \supervisor{Designing, implementing, and studying processes of crowdsourced policymaking}
  \description{
  	- Applied exploratory data analysis techniques to study the profile of the participants of crowdsourced policy-making processes\\
  	- Employed non-parametric statistical tests (Wilcox, Spearman, Chi-square, Kruskal-Wallis, Friedman) to examine change in the motivation factors that drive people to crowdsourced civic participation processes\\
	- Used Logistic Regression to predict the odds of participants to stay engaged in crowdsourcing processes\\
	- \cusemph{Co-authored four scientific publications on the topics of Crowdsourcing for Democracy, Collective Intelligence, and Machine Learning}\\\\
	\footnotesize{\textcolor{mygray}{\textbf{Research methods applied:} surveys}}
  }
\end{project}
\begin{project}
  \title{The California Report Card}
  \supervisor{Enhancing communication between elected authorities and the public}
  \description{
	- Modeled, designed, and implemented a Twitter app that allows citizens of California to propose suggestions on issues that merit the attention of the government\\
	- Contributed to the execution of a real-case process of crowdsourced policymaking in the state of California ($\sim$ 10,000 participants)\\
	- \cusemph{Co-authored a scientific publication on the topic of Crowdsourcing for Democracy}\\\\
	\footnotesize{\textcolor{mygray}{\textbf{Research methods applied:} prototyping}}
  }
\end{project}
\begin{project}
  \title{Agora 2.0}
  \supervisor{Enhancing Civic Participation through Public Displays}
  \description{
	- Designed and developed a platform that integrates a crowdsourcing idea system with public displays\\
	- Conducted a real-case process of participatory public service innovation in the city of Trento, Italy\\ 
	- \cusemph{Co-authored a scientific publication on the topic of E-Participation}\\\\
	\footnotesize{\textcolor{mygray}{\textbf{Research methods applied:} field studies - observational studies - interviews}}
  }
\end{project}
\begin{project}
  \title{BPM4People}
  \supervisor{Business Process Modeling for Participatory Enterprises, Organizations, and Public Administration Bodies}
  \description{
	- Conducted a literature review study on methodologies, techniques, and approaches to constructing tools for modeling and deploying of business processes that will be executed collaboratively by people on social networks\\
	- Applied multivariate linear regression analysis to study effectiveness of current social sharing practices\\
	- \cusemph{Co-authored two scientific publications on the topic of E-Participation}\\\\
	\footnotesize{\textcolor{mygray}{\textbf{Research methods applied:} controlled experiments}}
  }
\end{project}
\begin{project}
  \title{Ianus}
  \supervisor{Platform for the Simplification, Re-organization and Improvement of Business Processes}
  \description{
	- Designed and developed of models and systems to improve the communication between business analysts and developers and facilitate the analysis of business processes\\
	%- Contributed to the design and execution of usability tests\\
	- \cusemph{Co-authored a scientific publication on topic Business Process Management}\\\\
	\footnotesize{\textcolor{mygray}{\textbf{Research methods applied:} usability tests - surveys - prototyping}}
  }
\end{project}
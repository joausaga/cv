\documentclass{simplecv}
\usepackage[latin1]{inputenc}
\usepackage[spanish]{babel}
\usepackage{hyperref}
\usepackage{url}

\begin{document}

\leftheader{\\
\texttt{jorgesaldivar@gmail.com}\\
%\texttt{\small \url{http://jorgesaldivargalli.com/portfolio}}
}

%\lefttheader{Avinguda de la Republica Argentina 261 5-2, Barcelona, Spain\\
%\texttt{\small \url{http://jorgesaldivargalli.com/portfolio}}
%}

\title{Jorge Saldivar Galli}

\maketitle

\section{Summary}

Information Technology Engineer and open-source supporter with experience as a software developer, researcher and teacher assistant.

\section{Interests}

Social Impact technology, Democratic Innovation, Collective Intelligence, Open Innovation, Crowdsourcing, Software Engineering, Business Process Management, and Social Network Analysis.

\section{Education}

\begin{topic}

\item[2011 - current] {\bf PhD in Information and Communication Technology} - ICT Doctoral School, University of Trento, Italy.

\item[2001 - 2007] {\bf Bachelor of Science (B.Sc.), Informatics Engineering} - Catholic University ''Nuestra Se�ora de la Asunci�n'', Asunci�n, Paraguay. 

{\bf Thesis:} \emph{Improving cooperative and competitive behaviors on visually impaired children}. 

\end{topic}

\section{Awards and Honors}

\begin{topic}

\item PhD On the Move scholarship for PhD students at the University of Trento, Italy.


\item FC-UPM-IB scholarship for post-graduated studies at Universidad Polit�cnica de Madrid.

\item Golden-medal for best graduated student of 2000 class at San Jos� High-School.

\end{topic}

\section{Professional Experience}

\begin{topic}

\item[July 2014 - Now] \emph{Visiting Scholar} - Center for Information Technology Research in Interest of the Society (CITRIS), School of Electrical Engineering and Computer Science, University of California, Berkeley, California, USA.

Implementation of an application that enables public institutions to stay informed about public opinions and priorities posted on Twitter.

\textbf{Platform}: Web-based.

\textbf{Technologies}: Python, Django, Twitter API.

\textbf{Repository}: \url{https://github.com/joausaga/participa}

\item[October 2013 - December 2013] \emph{Visiting Researcher} - Information Analysis Division Hewlett-Packard Laboratories, Palo Alto, California, USA.

Modeling, designing and implementing analytic tools for large-scale parallel-executed SQL queries.

\textbf{Platform}: Web-based.

\textbf{Technologies}: Javascript, Java.

\item[November 2011 - August 2013] \emph{Researcher} - BPM4People Project (\url{http://www.bpm4people.org}): EU-funded research project.

BPM4People aims at constructing simple, low-cost, high performance methodology, tool suite, and application portfolio that will help B2B organizations harness the flexibility of social business processes design and deployment.

\item[November 2011 - March 2013] \emph{Researcher} - Ianus Project (\url{https://sites.google.com/site/ssbptester}): Trento-province-funded research project.

IANUS aims at business process improvement through a business activity monitoring (BAM) solution that is able to monitor the execution of business processes in highly distributed and loosely-coupled environments.

Development of a spreadsheet-based business process testing tool.

\textbf{Platform}: Web-based.

\textbf{Technologies}: Javascript, Java, Google App Scripts, Google Drive API, MySQL, Activiti.

\textbf{Repository}: \url{http://goo.gl/ZaVlwg}

\item[January 2010 - July 2011] \emph{Head of Education Technology} - NGO Paraguay Educa.

Responsible for the deployment of innovative technologies in schools benefited the project One Laptop per Child. 

Development of applications and main-stream patches.

\textbf{Platform}: Sugar Learning Platform

\textbf{Technologies}: Python, PHP, Javascript, MySQL, Git.

\textbf{Repository}: \url{http://git.sugarlabs.org/~jasg}

\item[March 2008 - December 2008] \emph{Head of Technology} - One Laptop per Child pilot project

Responsible of the technical aspects needed for deploying the pilot. Mounting servers, setting up XO laptops, developing sugar learning platform applications.

\textbf{Platform}: Sugar Learning Platform

\textbf{Technologies}: Python.

\item[March 2007 - December 2007] \emph{Software Developer} - Azucarera Iturbe: Sugar Industry

Development, test and maintenance of a business management system.

\textbf{Platform}: Web-based.

\textbf{Technologies}: Ruby on Rails, Javascript.

\item [February 2006 - July 2007] \emph{Assistant Professor} - Science and Technology School, Catholic University ``Nuestra Se�ora de la Asunci�n''

Assistant Professor in Programming Languages 1 class of the Informatics Engineering program.

\item [December 1999 - December 1999] \emph{Software Engineer} - High-School internship at Casa Escauriza I.C.S.A: Cane Industry

Development of a billing system.

\textbf{Platform}: Win 32.

\textbf{Technologies}: Visual Basic.

\end{topic}

\section{Technical skills}

\begin{itemize}

\item Languages: Java, Python, Javascript, R, PHP, SQL, Ruby, C, C++, Bash, $C\sharp$, Visual Basic.

\item Frameworks: Ruby on Rails, Yii, ExtJS, Qooxdoo, Drupal, Play, JQuery, Bootstrap, Django.

\item API: Twitter, Google Translator, Google Drive.

\item Operating Systems: Linux, OS X, Windows.

\end{itemize}

\section{Open source contributions}

\begin{topic}

\item \emph{Club de Othello}: Educational application for visually impaired children (\url{http://activities.sugarlabs.org/en-US/sugar/addon/4286})

\item \emph{Poll}: Application for creating polls, collecting votes, and analyzing the community opinions. (\url{http://activities.sugarlabs.org/en-US/sugar/addon/4074})

\item \emph{Labyrinth}: Mind-mapping application that supports text, images, and simple drawings. (\url{http://activities.sugarlabs.org/en-US/sugar/addon/4078})

\item \emph{Fingerprint Attendance}: Desktop software application that leverages on fingerprints the attendance at events, classes or work. (\url{http://sourceforge.net/projects/attendencesys})

\item \emph{Agora 2.0}: Web-based platform for fostering on-line and on-site civic-participation. (\url{http://github.com/joausaga/agora20})

\end{topic}

%\section{Publications in Refereed Journals}

%\begin{thebibliography}{10}

%\end{thebibliography}

\section{Publications in Refereed Conferences and Journals}

\begin{thebibliography}{10}

\footnotesize

\bibitem{aitamurto2014}
[UNDER-REVIEW] T. Aitamurto, J. Saldivar, J. Salminen. \emph{Self-selection in Crowdsourcing Democracy: A Bug or A Future?}. The Morphing Organization Workshop, ACM Conference on Supporting Groupwork, 2014, Sanibel Island, Florida, USA.

\bibitem{sal14s}
[UNDER-REVIEW] J. Saldivar, C. Vairetti, C. Rodriguez, F. Daniel, F. Casati, R. Alarc�n. \emph{Spreadsheet-based testing of business process}. Elsevier Information Systems.

\bibitem{sal14pdc}
J. Saldivar, C. Parra, C. Rodriguez, L. Cernuzzi, V. D'Andrea. \emph{Participa: fostering civic participation for public services innovation}. 13th Participatory Design Conference, 2014, Windhoek, Namibia.

\bibitem{sch12c}
G. Schiavo, M. Milano, J. Saldivar, T. Nasir, M. Zancanaro, G. Convertino. \emph{Agora2.0: Enhancing Civic Participation through a Public Display}. 6th International Conference on Communities and Technologies, 2013, Munich, Germany.

%\bibitem{sal13iar1}
%[TECHREPORT] J.Saldivar, C. Rodr�guez, F. Daniel, F. Casati. \emph{Rapporto su %tecniche di simulazione dei servizi attivati dai processi}. Ianus Project, 2013.

%\bibitem{sal13iar2}
%[TECHREPORT] J.Saldivar, F. Daniel, C. Vairetti, C. Rodr�guez, F. Casati. \emph{Rapporto su tecniche di esecuzione di functional regression sui processi}. Ianus Project, 2013.

%\bibitem{sal13iar3}
%[TECHREPORT] C. Vairetti, C. Rodr�guez, J.Saldivar, F. Daniel, F. Casati. \emph{Rapporto sui metodi innovativi di verifica della correttezza funzionale}. Ianus Project, 2013.

\bibitem{sal10i}
J. Saldivar, L. Cernuzzi. \emph{Resultados del Club de Othello XO: Una Experiencia de Aprendizaje en Ambiente de Interacci�n Social}. Congress of Technology Support Disabilities (IBERDISCAP), 2010, Ciudad de Mexico, Mexico.

\bibitem{cer08i}
L. Cernuzzi, J. Saldivar.\emph{Club de Othello Xo: Una Experiencia de Aprendizaje en Ambiente de Interacci�n Social}. Congress of Technology Support Disabilities (IBERDISCAP), 2008, Cartegena, Colombia.

\end{thebibliography}

%\section{References:}

%\begin{itemize}

%\item Phd. Luca Cernuzzi, Dean, Science and Technology Faculty, Catholic University "Nuestra Se�ora de la Asunci�n" . Phone: (+595-21) 334.650. Fax: (+595-21) 310-072. Email: lcernuzz@uca.edu.py.

%\end{itemize}

%\section{Additional Information}

%I am an Engineer very passionated about computer science and technology, self-motivated and hard-worker. I love to design and develop computer programs, solve hard problems and learn new algorithms, techniques and tools.

\end{document}

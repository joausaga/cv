\documentclass[helvetica,english,logo,notitle,totpages,utf8]{europecv2013}
\usepackage{graphicx}
\usepackage[a4paper,top=1.2cm,left=1.2cm,right=1.2cm,bottom=2.5cm]{geometry}
\usepackage[english]{babel}
\usepackage[T1]{fontenc}

%[Tutti i campi del CV sono facoltativi. Rimuovere i campi vuoti.]
\ecvname{Jorge Saldivar Galli}
\ecvaddress{Cesar Lopez Moreira 444, Asuncion, Paraguay}
\ecvtelephone{+585981534660}
\ecvemail{jorgesaldivar@gmail.com}
\ecvlinkedin{\href{http://www.linkedin.com/profile/view?id=160370332}{www.linkedin.com/profile/view?id=160370332}}
\ecvgender{Male}
\ecvdateofbirth{21/12/1981}
\ecvnationality{Paraguayan}

\ecvfootnote{© European Union, 2002-2015 | http://europass.cedefop.europa.eu}
%\ecvbeforepicture{\raggedleft}
%\ecvpicture[width=2.5cm]{file-immagine-eps}
%\ecvafterpicture{\ecvspace{-37mm}}

\begin{document}
\selectlanguage{english}

\begin{europecv}
\ecvpersonalinfo[10pt]

\ecvposition{Job applied for
Position
Preferred job
Studies applied for}{%Study grant for doctoral students not entitled to doctoral grant and enrolled to the ICT International Doctoral School, at the XXVII edition, of the University of Trento. Decree no. 162 dated 13th April 2015.
}

\ecvsection{Work experience}
%[Add separate entries for each experience. Start from the most recent.]

\ecvworkexperience{February 2016 - Present}{Assistant Researcher}{Participa Project}{Catholic University of Asuncion, Paraguay}{Research, modeling, design and implementation of IT applications and services that support and facilitate the participation of Paraguayan citizens in innovating public services.}

%\ecvworkexperience{Researcher{March 2014 - Present}}{The Finnish Experiment Project}{Democratic innovation project funded by the Committee for the Future of the Parliament of Finland.}{Study on how collective intelligence, whether gathered by crowdsourcing and/or principles of open innovation, impacts processes of policy making.}

%\ecvworkexperience{February 2015 - September 2015}{Teaching Assistant}{}{University of Trento, Italy.}{Teaching Assistant for the undergraduate course on Web Programming Technologies. The course is focuses on J2EE web technologies (e.g., Servlet, JSP, JSTL, JavaBean) and the work involves mainly the confection of online teaching materials and the preparation of tests.}

\ecvworkexperience{July 2014 - October 2014}{Visiting Scholar}{California Report Card Project}{University of California, Berkeley. Center for Information Technology Research in Interest of the Society (CITRIS), Berkeley, California, USA.}{Modeled, designed, and implemented a Twitter-based extension for the platform CAFE used in the project The California Report Card to engage citizens of California in the collective assessment of timely issues.}

\ecvworkexperience{October 2013 - December 2013}{Visiting Researcher}{}{Hewlett-Packard Laboratories, Information Analysis Division, Palo Alto, California, USA.}{Modeled, designed and implemented an web-based analytic tool for conducting real-time visualization of query execution. It converts a textual query execution plan to a tree visualization and as query executes, its tree visualization is annotated with live query execution events as for example, progress and live statistics for each query operator, overall query progress, etc. 

Developed a recommendation plugin that annotates the query tree with the most interesting results of its execution analysis, which may indicate execution problems and recommend design solutions for problem queries.}

\ecvworkexperience{November 2011 - August 2013}{Assistant Researcher}{BPM4People Project}{EU FP7 Research Project.}{Researched about methodologies, techniques, and approaches to construct tools for modeling and deploying of business processes that will be executed collaboratively by people in social networks.}

\ecvworkexperience{November 2011 - March 2013}{Assistant Researcher}{Ianus Project}{Trento-province-funded research project.}{Modeled, designed and implemented a spreadsheet-based tool for enabling business process (BP) analysts to conduct BP testing tasks, such as writing metrics and assertions, running tests, and designing test report.

Designed and executed usability studies.}

\ecvworkexperience{January 2010 - July 2011}{Head of Education Technology}{NGO Paraguay Educa.}{}{Developed and maintenance of Sugar OS apps (Labyrinth, Poll, TamTam). 

Designed, modeled, implemented and maintained the educational software stack (learning platforms, online libraries, student tracking systems) used by teachers and students in classrooms during the implementation of the project One Laptop per Child (OLPC).}

%\ecvworkexperience{March 2008 - December 2008}{Head of Technology}{One Laptop per Child pilot project}{Catholic University ``Nuestra Señora de la Asunción'', Ministry of Education and Culture of Paraguay, Inter-American Development Bank, Organization of Inter-American States.}{Mounted and maintained of Unix servers. Set up and configured OLPC XO laptops. Developed and customized educational applications for Sugar OS.}

%\ecvworkexperience{March 2007 - December 2007}{Software Developer}{Azucarera Iturbe: Sugar Industry}{}{Developed, tested and maintained a web-based (ruby on rails) business management system (billing, stocking, accounting, freighting, human resources)}

%\ecvworkexperience{}{}{}{}{}

\ecvsection{Teaching Experience}

\ecvworkexperience{February 2016 - Present}{Cotutor of Bachelor Final Projects}{}{Catholic University of Asuncion, Paraguay.}{Titles: Context-aware civic participation and Automatic processing of civic contributions through Machine Learning and Data Visualization.}

\ecvworkexperience{February 2015 - September 2015}{Teaching Assistant}{Introduction to Web Programming}{Department of Information Engineering and Computer Science, University of Trento, Italy.}{Teaching Assistant for the undergraduate course on Web Programming Technologies. The course is focuses on J2EE web technologies (e.g., Servlet, JSP, JSTL, JavaBean) and the work involves mainly the confection of online teaching materials and the preparation of tests.}

\ecvworkexperience{November 2014 - March 2015}{Master Thesis Cotutor}{}{Department of Information Engineering and Computer Science, University of Trento, Italy.}{Shahadat Chowdhury's Master Thesis Co-Supervisor. Title: Analyzing and Visualizing citizen opinions collected from social networks.}

\ecvworkexperience{May 2010 - July 2010}{Lecturer}{Sugar: development of an educational platform using open source code}{Polytechnic School, National University of Asuncion, Paraguay.}{Tutor of a course about development for Sugar OS.}

\ecvworkexperience{October 2008 - December 2008}{Training Facilitator}{}{NGO Paraguay Educa}{Training of trainers within the Paraguayan deployment of One Laptop Per Child project.}

\ecvworkexperience{February 2007 - July 2007}{Teaching Assistant}{Programming Languages 1}{Science and Technology School, Catholic University ``Nuestra Señora de la Asunción''}{Teaching Assistant in Programming Languages 1 part of the System Analysis program.}

\ecvworkexperience{August 2006 - December 2006}{Teaching Assistant}{Programming Languages 1}{Science and Technology School, Catholic University ``Nuestra Señora de la Asunción''}{Teaching Assistant in Programming Languages 1 part of the Information Technology Engineering program.}

\ecvworkexperience{February 2006 - July 2006}{Teaching Assistant}{Programming Languages 1}{Science and Technology School, Catholic University ``Nuestra Señora de la Asunción''}{Teaching Assistant in Programming Languages 1 part of the Information Technology Engineering program.}

\ecvsection{Education and training}
%[Add separate entries for each course. Start from the most recent.]

\ecveducation{2011 - 2016}{PhD in Information and Communication Technology}{ICT Doctoral School, University of Trento, Italy.}{Thesis: Empowering Online Idea Management through Public Displays and Social Networking Services}{}

\ecveducation{2001 - 2006}{Bachelor of Science (B.Sc.), Informatics Engineering}{Catholic University ''Nuestra Señora de la Asunción'', Asunción, Paraguay.}{Thesis: Improving cooperative and competitive behaviors on visually impaired children}{}

\ecvsection{Personal skills}

\ecvmothertongue[20pt]{Spanish}
\ecvlanguageheader
\ecvlanguage{English}{C1}{C1}{B2}{B2}{B2}
\ecvlanguage{Italian}{B1}{B1}{B1}{B1}{A2}
\ecvlastlanguage{Portuguese}{C1}{B2}{A2}{A2}{A1}
\ecvlanguagefooter[10pt]

\ecvitem[10pt]{Communication skills}{Good communication skills gained through my experience as coordinator and public speaker in retreats, seminars, and courses.}

\ecvitem[10pt]{Organizational / managerial skills}{Excellent teamwork skills thanks to my experience as:
\begin{itemize}
\item Coordinator of social campaigns to assist families of urban areas in financial hardship.
\item Electoral delegate in local and national elections. 
\item Over 11 years of experience leading teams and coordinating medium-scale seminars, retreats, courses, and talks.
\item Four years as member of the directory board of the association ``Movimiento Peregrino''.
\item Two year as general coordinator of the directory board of the association ``Movimiento Peregrino''.
\end{itemize}
}

\ecvitem[10pt]{Job-related skills}{Enough experience as teaching assistant and course facilitator thanks to my previous work at the Catholic University ``Nuestra Señora de la Asunción'', Paraguay, at the NGO Paraguay Educa, and at the Polytechnic School, National University of Asuncion, Paraguay.}

\ecvitem[10pt]{Computer skills}{Proficient in the programming languages: Java, Python, Javascript, R, PHP, SQL, Ruby, C, C++, Bash, $C\sharp$, Visual Basic; in the development frameworks: Ruby on Rails, Yii, ExtJS, Qooxdoo, Drupal, Play, JQuery, Bootstrap, Django; and in the operating systems: Linux, OS X.  Proficient in the e-learning tools: Moodle, Proprofs, Claroline.}

%\ecvitem[10pt]{Other skills}{Replace with other relevant skills not already mentioned. Specify in what context they were acquired. Example:\par
%carpentry}
%\ecvitem{Driving licence}{Replace with driving licence category/-ies. Example:\par
%B}

\ecvsection{Additional information}

\ecvitem[10pt]{Publications}{
\begin{itemize}
\item Gianluca Schiavo, Marco Milano, Jorge Saldivar, Tooba Nasir, Massimo Zancanaro, and Gregorio Convertino. Agora 2.0: Enhancing civic participation through a public display. In Proceedings of the \textit{6th International Conference on Communities and Technologies (C\&T), pp. 46-54}. ACM, 2013.
\item Jorge Saldivar, Cristhian Parra, Carlos Rodríguez, Luca Cernuzzi, and Vincenzo D’Andrea. Participa: Fostering civic participation for public services innovation. In \textit{13th Participatory Design Conference}. 2014.
\item Tanja Aitamurto, Jorge Saldivar, and Juho Salminen. Self-selection In Crowdsourced Democracy: A Bug Or A Feature?. In \textit{GROUP Conference}. ACM, 2014.
\item Jorge Saldivar, Carla Vairetti, Carlos Rodríguez, Florian Daniel, Fabio Casati, and Rosa Alarcón. Analysis and improvement of business process models using spreadsheets. In \textit{Information Systems 57: 1-19}. Elsevier, 2016.
\item Jorge Saldivar, Carlos Rodriguez, Florian Daniel, Fabio Casati, and Luca Cernuzzi. On the (in)effectiveness of the Share/Tweet button: A study in the context of idea management for civic participation. In \textit{IEEE Internet Computing, pp. 1, 5555}. IEEE, 2016.
\item Tanja Aitamurto, Hélène Landemore, and Jorge Saldivar. Unmasking the crowd: participants’ motivation factors, expectations, and profile in a crowdsourced law reform. In \textit{Information, Communication \& Society: 1-22}. 2016.
\item Tanja Aitamurto, Kaiping Chen, Ahmed Cherif, Jorge Saldivar, and Luis Santana. Civic CrowdAnalytics: Making sense of crowdsourced civic input with big data tools. In Proceedings of the \textit{20th International Academic Mindtrek Conference, pp. 86-94}. ACM, 2016. (\textbf{\textit{Best paper award}}).
\item Jorge Saldivar, Florian Daniel, Fabio Casati, and Luca Cernuzzi. Idea Management in Social Networks: A Study of how to Tap into the Ideas of Facebook Communities. In Proceedings of the \textit{17th International Conference on Collaboration Technologies and Systems (CTS), pp. 3-10}. IEEE, 2016.
\item Jorge Saldivar, Marcos Báez, Carlos Rodríguez, Gregorio Convertino, and Grzegorz Kowalik. Idea Management Communities in the Wild: An exploratory study of 166 online communities. In Proceedings of the \textit{17th International Conference on Collaboration Technologies and Systems (CTS), pp. 81-89}. IEEE, 2016.
\item Tanja Aitamurto and Jorge Saldivar. Examining the Quality of Crowdsourced Deliberation: Respect, Reciprocity and Lack of Common-Good Orientation. \textit{Proceedings of the 2017 CHI Conference Extended Abstracts on Human Factors in Computing Systems}. ACM, 2017.
\item Jorge Saldivar, Cristhian Parra, Marcelo Alcaraz, Rebeca Arteta, and Luca Cernuzzi. Civic Technologies for Social Innovation: A Systematic Literature Review \textit{(submitted)}. In \textit{Computer Supported Cooperative Work Journal (CSCWJ)}. 2017.
\item Tanja Aitamurto and Jorge Saldivar. The Irrationality Rational and Quickly-Satisfied Crowd: Motivational Factors in Crowdsourced Policymaking \textit{(submitted)}. In \textit{Computer Supported Cooperative Work Conference}. 2018.
\end{itemize}
}

\ecvitem[10pt]{Honours and awards}{
\begin{itemize}
\item Trento RISE association PhD on the move scholarship. Trento, Italy (2013).
\item University of Trento three-years PhD student fellowship. Trento, Italy (2011).
%\item FC-UPM-IB scholarship for post-graduated studies at Universidad Politécnica de Madrid, Spain (2011).
\item Golden-medal for best graduated student of 2000 class at San Jose High-School, Asuncion, Paraguay (2000).
\end{itemize}
}

\ecvitem[10pt]{Patents}{
\begin{itemize}
\item A. Simitsis, W. K. Wilkinson, J. Saldivar. 2013. A method for monitoring and analyzing in real-time the execution of large-scale parallel-executed SQL queries. U.S. Patent Application 83822125, filed January 2014. Patent Pending.
\end{itemize}
}

\ecvitem[10pt]{Outcomes of research projects}{
\begin{itemize}
\item \emph{Participa}: Civic participation platform support the execution of public consultants, idea campaigns, and opinion polls directly in hashtag-supported social networking sites, such as Twitter, Facebook, and Google Plus. (\url{https://github.com/joausaga/participa})
\item \emph{Agora 2.0}: Civic participation oriented tool for polling citizens' opinions regarding local public interest issues in a synchronized on-line and on-site setting. (\url{http://github.com/joausaga/agora20})
\item \emph{Club de Othello}: Educational tool for helping visually impaired children to learn spatio-temporal concepts as well as to allow them to cooperate and compete with their peers (\url{http://activities.sugarlabs.org/en-US/sugar/addon/4286})
\item \emph{Spreadsheet-based BP testing}: Spreadsheet-based tool for enabling business process (BP) analysts to conduct BP testing tasks, such as writing metrics and assertions, running tests, and designing test report (\url{https://sites.google.com/site/ssbptester})
\end{itemize}
}

\ecvitem[10pt]{Open-source contributions}{
\begin{itemize}
\item \emph{IdeaScaly}: Python-based IdeaScale API client (\url{https://github.com/jouasaga/ideascaly})
\item \emph{Tweepy}: Python-based client for accessing Twitter API (\url{https://github.com/tweepy/tweepy})
\item \emph{Poll}: OLPC Sugar application for creating polls, collecting votes, and analyzing the community opinions. (\url{http://activities.sugarlabs.org/en-US/sugar/addon/4074})
\item \emph{Labyrinth}: OLPC Sugar mind-mapping application that supports text, images, and simple drawings. (\url{http://activities.sugarlabs.org/en-US/sugar/addon/4078})
\item \emph{TamTam}: OLPC Sugar application that allows children in exploring music instruments. (\url{http://activities.sugarlabs.org/en-US/sugar/addon/4061})
\item \emph{Dextrose}: Distribution of OLPC Sugar operating system jointly developed by Activity Central, Paraguay Educa, and Plan Ceibal (\url{http://wiki.sugarlabs.org/go/Dextrose})
\end{itemize}
}

%\ecvitem[10pt]{
%Presentations
%Projects
%Conferences
%Seminars
%Memberships
%References}{Replace with relevant publications, presentations, projects, conferences, seminars, honours and awards, memberships, references. Remove headings not relevant in the left column.}

%\ecvsection{Annexes}

%\ecvitem[10pt]{}{Replace with list of documents annexed to your CV. Examples:
%\begin{itemize}
%\item copies of degrees and qualifications; 
%\item testimonial of employment or work placement;
%\item publications or research.
%\end{itemize}}

\end{europecv}
\end{document} 
\documentclass{simplecv}
\usepackage[latin1]{inputenc}
\usepackage[spanish]{babel}
\usepackage{hyperref}
\usepackage{url}

\begin{document}

\leftheader{\\
\texttt{\small{Via delle Cave 13, Trento, Italy}}\\
\texttt{\small{jorgesaldivar@gmail.com}}
}

%\lefttheader{\\
%\texttt{Via delle Cave 13, Trento, Italy}
%}

\title{Jorge Saldivar Galli}
\maketitle

\section{Summary}

P.h.D. student, informatics engineer, and open-source supporter with experience as a software developer, researcher and teacher assistant.

\section{Interests}

Social Impact technology, Democratic Innovation, Collective Intelligence, Open Innovation, Crowdsourcing, Software Engineering, and Social Network Analysis.

\section{Education}

\begin{topic}

\item[2011 - current] {\bf PhD in Information and Communication Technology} - ICT Doctoral School, University of Trento, Italy.

\item[2001 - 2007] {\bf Bachelor of Science (B.Sc.), Informatics Engineering} - Catholic University ''Nuestra Se�ora de la Asunci�n'', Asunci�n, Paraguay. 

{\bf Thesis:} \emph{Improving cooperative and competitive behaviors on visually impaired children}. 

\end{topic}

\section{Professional Experience}

\begin{topic}

\item[July 2014 - October 2014] \emph{Visiting Scholar} - University of California, Berkeley. Center for Information Technology Research in Interest of the Society (CITRIS), Berkeley, California, USA.

Modeled, designed, and implemented a civic-participation tool for supporting the execution of public consultants, idea campaigns, and opinion polls directly in hashtag-supported social networking sites, such as Twitter, Facebook, and Google Plus. The tool was deployed as an extension of the platform CAFE used in the project California Report Card to engage citizens of California in the collective assessment of timely issues.

\textbf{Platform}: Web.

\textbf{Technologies}: Python, Django.

\item[October 2013 - December 2013] \emph{Visiting Researcher} - Information Analysis Division Hewlett-Packard Laboratories, Palo Alto, California, USA.

Modeled, designed and implemented an web-based analytic tool for conducting real-time visualization of query execution. It converts a textual query execution plan to a tree visualization and as query executes, its tree visualization is annotated with live query execution events as for example, progress and live statistics for each query operator, overall query progress, etc. Also developed a recommendation plugin that annotates the query tree with the most interesting results of its execution analysis, which may indicate execution problems and recommend design solutions for problem queries.

\textbf{Platform}: Web.

\textbf{Technologies}: Javascript, JQuery, Java.

\item[November 2011 - August 2013] \emph{Researcher} - BPM4People Project (\url{http://www.bpm4people.org}): EU-funded research project.

Researched about methodologies, techniques, and approaches to construct tools for modeling and deploying of business processes that will be executed collaboratively by people in social networks.

\item[November 2011 - March 2013] \emph{Researcher} - Ianus Project: Trento-province-funded research project.

Modeled, designed and implemented a spreadsheet-based tool for enabling business process (BP) analysts to conduct BP testing tasks, such as writing metrics and assertions, running tests, and designing test report.

\textbf{Platform}: Web.

\textbf{Technologies}: Play, Google App Scripts, Google Drive API, Activiti.


\item[January 2010 - July 2011] \emph{Head of Education Technology} - NGO Paraguay Educa.

Developed OLPC Sugar applications, implemented and maintained the educational software stack used by teachers and students in classrooms.

\textbf{Platforms}: OLPC XO laptops, Sugar Operating System

\textbf{Technologies}: Python, Javascript.

\textbf{Repository}: \url{http://git.sugarlabs.org/~jasg}

\item[March 2008 - December 2008] \emph{Head of Technology} - One Laptop per Child pilot project

Deployed the technical setting needed for running a pilot project of OLPC. Among others tasks: mounted servers, set up XO laptops, and developed applications for Sugar OS.

\textbf{Platforms}: OLPC XO laptops, Sugar Operating System

\textbf{Technologies}: Python.

\item[March 2007 - December 2007] \emph{Software Developer} - Azucarera Iturbe: Sugar Industry

Developed, tested and maintained a business management system.

\textbf{Platform}: Web.

\textbf{Technologies}: Ruby on Rails, Qooxdoo.

%\item [February 2006 - July 2007] \emph{Assistant Professor} - Science and Technology School, Catholic University ``Nuestra Se�ora de la Asunci�n''

%Teaching Assistant in Programming Languages 1 class of the Informatics Engineering program.

\item [December 1999 - December 1999] \emph{Software Engineer} - High-School internship at Casa Escauriza I.C.S.A: Cane Industry

Developed a billing system.

\textbf{Platform}: Win 32.

\textbf{Technologies}: Visual Basic.

\end{topic}

\section{Technical skills}

\begin{itemize}

\item Languages: Java, Python, Javascript, R, PHP, SQL, Ruby, C, C++, Bash, $C\sharp$, Visual Basic.

\item Frameworks: Ruby on Rails, Yii, ExtJS, Qooxdoo, Drupal, Play, JQuery, Bootstrap, Django.

\item API: Twitter, Google Translator, Google Drive, Google URL shortener.

\item Operating Systems: Linux, OS X, Windows.

\end{itemize}

\section{Awards and Honors}

\begin{topic}

\item Trento RISE association PhD on the move scholarship. Trento, Italy (2013).

\item University of Trento three-years PhD student fellowship. Trento, Italy (2011).

\item FC-UPM-IB scholarship for post-graduated studies at Universidad Polit�cnica de Madrid, Spain (2011).

\item Golden-medal for best graduated student of 2000 class at San Jose High-School, Asuncion, Paraguay (2000).

\end{topic}

\section{Outcomes of research projects}

\begin{itemize}

\item \emph{Participa}: Civic participation platform support the execution of public consultants, idea campaigns, and opinion polls directly in hashtag-supported social networking sites, such as Twitter, Facebook, and Google Plus. (\url{https://github.com/joausaga/participa})

\item \emph{Agora 2.0}: Civic participation oriented tool for polling citizens' opinions regarding local public interest issues in a synchronized on-line and on-site setting. (\url{http://github.com/joausaga/agora20})

\item \emph{Club de Othello}: Educational tool for helping visually impaired children to learn spatio-temporal concepts as well as to allow them to cooperate and compete with their peers (\url{http://activities.sugarlabs.org/en-US/sugar/addon/4286})

\item \emph{Spreadsheet-based BP testing}: Spreadsheet-based tool for enabling business process (BP) analysts to conduct BP testing tasks, such as writing metrics and assertions, running tests, and designing test report (\url{https://sites.google.com/site/ssbptester})

\end{itemize}

\section{Open-source contributions}

\begin{itemize}

\item \emph{Tweepy}: Python-based client for accessing Twitter API (\url{https://github.com/tweepy/tweepy})

%\item \emph{Khan Excercises}: Khan Academy framework for building exercises (\url{https://github.com/Khan/khan-exercises})

\item \emph{Poll}: OLPC Sugar application for creating polls, collecting votes, and analyzing the community opinions. (\url{http://activities.sugarlabs.org/en-US/sugar/addon/4074})

\item \emph{Labyrinth}: OLPC Sugar mind-mapping application that supports text, images, and simple drawings. (\url{http://activities.sugarlabs.org/en-US/sugar/addon/4078})

\item \emph{TamTam}: OLPC Sugar application that allows children in exploring music instruments. (\url{http://activities.sugarlabs.org/en-US/sugar/addon/4061})

\item \emph{Dextrose}: Distribution of OLPC Sugar operating system jointly developed by Activity Central, Paraguay Educa, and Plan Ceibal (\url{http://wiki.sugarlabs.org/go/Dextrose})

%\item \emph{Fingerprint Attendance}: Desktop software application that leverages on fingerprints the attendance at events, classes or work. (\url{http://sourceforge.net/projects/attendencesys})

\end{itemize}

%\section{Publications in Refereed Journals}

%\begin{thebibliography}{10}

%\end{thebibliography}

\section{Patents}

\begin{thebibliography}{10}

\footnotesize

\bibitem{sal14s}
A. Simitsis, W. K. Wilkinson, J. Saldivar. 2013. A method for monitoring and analyzing in real-time the execution of large-scale parallel-executed SQL queries. U.S. Patent Application 83822125, filed January 2014. Patent Pending.

\end{thebibliography}

\section{Publications in Refereed Conferences and Journals}

\begin{thebibliography}{10}

\footnotesize

\bibitem{sal15ic}
J. Saldivar, C. Rodriguez, F. Daniel, F. Casati, L. Cernuzzi. \emph{On the (in)effectiveness of the Share/Tweet button: A study in the context of idea management for civic participation}. IEEE Internet Computing Magazine. (under review)

\bibitem{sal14jou}
J. Saldivar, C. Vairetti, C. Rodriguez, F. Daniel, F. Casati, R. Alarc�n. \emph{Spreadsheet-based testing of business process}. Elsevier Information Systems Journal. (under review)

\bibitem{aitamurto2014}
T. Aitamurto, J. Saldivar, J. Salminen. \emph{Self-selection in Crowdsourcing Democracy: A Bug or A Future?}. Crowdwork Workshop, ACM Conference on Supporting Groupwork, 2014, Sanibel Island, Florida, USA.

\bibitem{sal14pdc}
J. Saldivar, C. Parra, C. Rodriguez, L. Cernuzzi, V. D'Andrea. \emph{Participa: fostering civic participation for public services innovation}. 13th Participatory Design Conference, 2014, Windhoek, Namibia.

\bibitem{sch12c}
G. Schiavo, M. Milano, J. Saldivar, T. Nasir, M. Zancanaro, G. Convertino. \emph{Agora2.0: Enhancing Civic Participation through a Public Display}. 6th International Conference on Communities and Technologies, 2013, Munich, Germany.

%\bibitem{sal13iar1}
%[TECHREPORT] J.Saldivar, C. Rodr�guez, F. Daniel, F. Casati. \emph{Rapporto su %tecniche di simulazione dei servizi attivati dai processi}. Ianus Project, 2013.

%\bibitem{sal13iar2}
%[TECHREPORT] J.Saldivar, F. Daniel, C. Vairetti, C. Rodr�guez, F. Casati. \emph{Rapporto su tecniche di esecuzione di functional regression sui processi}. Ianus Project, 2013.

%\bibitem{sal13iar3}
%[TECHREPORT] C. Vairetti, C. Rodr�guez, J.Saldivar, F. Daniel, F. Casati. \emph{Rapporto sui metodi innovativi di verifica della correttezza funzionale}. Ianus Project, 2013.

\bibitem{sal10i}
J. Saldivar, L. Cernuzzi. \emph{Resultados del Club de Othello XO: Una Experiencia de Aprendizaje en Ambiente de Interacci�n Social}. Congress of Technology Support Disabilities (IBERDISCAP), 2010, Ciudad de Mexico, Mexico.

\bibitem{cer08i}
L. Cernuzzi, J. Saldivar.\emph{Club de Othello Xo: Una Experiencia de Aprendizaje en Ambiente de Interacci�n Social}. Congress of Technology Support Disabilities (IBERDISCAP), 2008, Cartegena, Colombia.

\end{thebibliography}

%\section{References:}

%\begin{itemize}

%\item Phd. Luca Cernuzzi, Dean, Science and Technology Faculty, Catholic University "Nuestra Se�ora de la Asunci�n" . Phone: (+595-21) 334.650. Fax: (+595-21) 310-072. Email: lcernuzz@uca.edu.py.

%\end{itemize}

%\section{Additional Information}

%I am an Engineer very passionated about computer science and technology, self-motivated and hard-worker. I love to design and develop computer programs, solve hard problems and learn new algorithms, techniques and tools.

\end{document}

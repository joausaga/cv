% LaTeX resume using res.cls
\documentclass[line,margin]{res} 
\usepackage{helvetica} % uses helvetica postscript font (download helvetica.sty)
%\usepackage{newcent}   % uses new century schoolbook postscript font 
\usepackage[spanish]{babel}
\usepackage[latin1]{inputenc}


\begin{document}

\name{Jorge Saldivar Galli}
% \address used twice to have two lines of address

\address{Via delle Cave 13, Trento, Italy}
\address{jorgesaldivar@gmail.com}

\begin{resume}
 
\section{OBJECTIVE} Position as a fellow in the 2015 Data Science for Social Good program. 
 
\section{EDUCATION} 
	{\sl PhD candidate,} Information and Communication Technology,\\ 
ICT Doctoral School, University of Trento, Italy (expected April, 2016).
                
	{\sl Bachelor of Science (B.Sc.),} Informatics Engineering,\\ 
    Catholic University "Nuestra Se\~nora de la Asunci\'on", Paraguay, 2010.\\
    {\bf Thesis:} \emph{Improving cooperative and competitive behaviors on visually impaired children}.
 
\section{SOCIAL IMPACT RESEARCH PROJECTS}
	{\sl Club de Othello:} Educational project that seeks for exploiting the expressiveness of multimedia tools to help visually-impaired children in learning spatio-temporal concepts and in improving their social abilities
	
	{\sl Agora 2.0:} Project that aims at combining the potential of public displays and the power of online platforms to create a synchronized on-line and on-site system oriented to promote the participation of citizens in discussions regarding public concern issues.
	
	{\sl Participa (PhD thesis):} Civic-participation project that aims at harvesting the expertise/knowledge/opinions of today's biggest virtual communities -- the social networks -- to drive, enrich, and complement the, increasingly common, internet-mediated public consultant initiatives.
	
	%{\sl The Finland Experiment:} Analysis on the participation patterns and the effect of moderation actions in the context of the Finland crowdsourced law-reform process.
 
\section{RESEARCH \\ EXPERIENCE} 
{\sl Visiting Scholar} \hfill July 2014 - October 2014 \\
Center for Information Technology Research in Interest of the Society (CITRIS), University of California, Berkeley, California, USA.
\begin{itemize}  \itemsep -2pt %reduce space between items
\item Modeled, designed, and implemented a Twitter-based extension for the platform CAFE used in the project California Report Card to engage citizens of California in the collective assessment of timely issues.
\end{itemize}

{\sl Visiting Researcher} \hfill October 2013 - December 2013 \\
Information Analysis Division, Hewlett-Packard (HP) Labs, Palo Alto, California, USA.
\begin{itemize}  \itemsep -2pt %reduce space between items
\item Designed and implemented an web-based analytic tool for conducting real-time visualization of query execution. In doing so, the query execution plan was converted into a visualization tree, which was annotated with live query execution progress statistics.
\item Developed a recommendation plugin that annotated the query tree with the most interesting results of its execution analysis, which may indicate execution problems and recommend design solutions for problem queries.
\end{itemize}

{\sl Researcher} \hfill November 2011 - October 2014 \\
EU-funded BPM4People research project and Trento-province-funded Ianus project.
\begin{itemize}  \itemsep -2pt %reduce space between items
\item Researched on methodologies, techniques, and approaches to construct tools for modeling and deploying of business processes that will be executed collaboratively by people in social networks.
\item Modeled, designed and implemented a spreadsheet-based tool for enabling business process (BP) analysts to conduct BP testing tasks, such as writing metrics and assertions, running tests, and designing test report.
\item Designed and executed usability studies.
\end{itemize}

\section{EMPLOYMENT} 
{\sl Software Developer/Head of Educational Technology} \hfill January 2010 - July 2011 \\
NGO Paraguay Educa. 
\begin{itemize}  \itemsep -2pt %reduce space between items
\item Developed and maintenance of Sugar OS apps (Labyrinth, Poll, TamTam).
\item Designed, modeled, implemented and maintained the educational software stack (learning platforms, online libraries, student tracking systems) used by teachers and students in classrooms during the implementation of the project One Laptop per Child (OLPC).
\end{itemize}
 
{\sl Software Developer/Technology Manager} \hfill        March 2008 - December 2008 \\
OLPC pilot project, Catholic University "Nuestra Se\~nora de la Asunci\'on" 
\begin{itemize}  \itemsep -2pt %reduce space between items
	\item Mounted and maintained of Unix servers. Set up and configured OLPC XO laptops. 
	\item Developed and customized educational applications for Sugar OS.
\end{itemize} 
     
{\sl Software Developer} \hfill        March 2007 - December 2007 \\
Azucarera Iturbe (Sugar Industry)
\begin{itemize}
	\item Developed, tested and maintained a web-based (ruby on rails) business management system (billing, stocking, accounting, freighting, human resources)
\end{itemize} 
 
\section{COMPUTER AND \\ DATA ANALYSIS \\ SKILLS} 
\begin{itemize} \itemsep -2pt      
\item Languages and Frameworks: Python, Java, Javascript, PHP, SQL, Ruby, C, C++, Bash, $C\sharp$, Ruby on Rails, Yii, ExtJS, Qooxdoo, Drupal, Play, JQuery, Bootstrap, Django.
\item Analysis Methods and Techniques: Descriptive statistics, correlation analysis, regression models, inferential statistics, grounded theory.
\item Data processing and visualization tools: R, Kettle, Ggplot, D3, Jit. 
%\item {\sl Web-development Frameworks:} 
%\item {\sl RESTful APIs:} Twitter, Google Plus, Facebook, IdeaScale, Google Translator, Google Drive, Google URL shortener.
    %\item {\sl Operating Systems:} OSX, Unix, Windows.
\end{itemize}

\section{EXTRA-CURRICULAR \\ ACTIVITIES}    
\begin{itemize} \itemsep -2pt      
	\item Foundation member of the non-for-profit Organization Paraguay Educa.
    \item Electoral delegate in local and national elections.
%    \item Team coordinator for medium-sized seminars, retreats, talks.
    \item Member of the directory board of the Pilgrim movement.
    \item Participant and coordinator of diverse social campaigns, e.g. {\sl Corazones Abiertos, Juventud que se mueve, SerVos}.
   \item Member of San Jos\'e high-school student association.
\end{itemize}

\section{AWARDS}
\begin{itemize} \itemsep -2pt 
	\item PhD student fellowship. University of Trento, Trento, Italy (2011-2014).
	\item PhD on the move scholarship. Trento RISE Association, Trento, Italy (2013).
	\item FC-UPM-IB scholarship for post-graduated studies at Universidad Polit\'ecnica de Madrid, Spain (2011).
	%\item Golden-medal for best graduated student of 2000 class at San Jos\'e High-School, Asuncion, Paraguay (2000).
\end{itemize}

\section{PUBLICATIONS}
\begin{itemize} \itemsep -2pt 
%	\item J. Saldivar, C. Rodriguez, F. Daniel, F. Casati, L. Cernuzzi. \emph{On the (in)effectiveness of the Share/Tweet button: A study in the context of idea management for civic participation}. IEEE Internet Computing Magazine. (under review)
%	\item J. Saldivar, C. Vairetti, C. Rodriguez, F. Daniel, F. Casati, R. Alarcon. \emph{Spreadsheet-based testing of business process}. Elsevier Information Systems Journal. (under review)
\item T. Aitamurto, J. Saldivar, J. Salminen. \emph{Self-selection in Crowdsourcing Democracy: A Bug or A Future?}. Crowdwork Workshop, ACM Conference on Supporting Groupwork, 2014, Sanibel Island, Florida, USA.
\item J. Saldivar, C. Parra, C. Rodriguez, L. Cernuzzi, V. D'Andrea. \emph{Participa: fostering civic participation for public services innovation}. 13th Participatory Design Conference, 2014, Windhoek, Namibia.
\item G. Schiavo, M. Milano, J. Saldivar, T. Nasir, M. Zancanaro, G. Convertino. \emph{Agora2.0: Enhancing Civic Participation through a Public Display}. 6th International Conference on Communities and Technologies, 2013, Munich, Germany.
\item J. Saldivar, L. Cernuzzi. \emph{Club de Othello XO: Una Experiencia de Aprendizaje en Ambiente de Interaccion Social}. Congress of Technology Support Disabilities (IBERDISCAP), 2010, Ciudad de Mexico, Mexico.
%\item L. Cernuzzi, J. Saldivar.\emph{Club de Othello Xo: Una Experiencia de Aprendizaje en Ambiente de Interaccion Social}. Congress of Technology Support Disabilities (IBERDISCAP), 2008, Cartegena, Colombia.
\end{itemize}

\end{resume}
\end{document}







